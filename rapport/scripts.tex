%\section{Logstash}

\section{Elasticsearch}
\begin{lstlisting}[style=code,label={lst:mappingresult},caption={Exemple de mapping}]
{
    "firewall-2015.06.10": {
        "mappings": { "syslog": {"properties": {
                    "@timestamp": {
                        "type": "date",
                        "format": "dateOptionalTime"
                    },
                    "@version": {
                    "type": "string"
                    },
                    "IPinside": {"type": "string"},
                    "IPnat": {"type": "string},
                    "IPoutside": {"type": "string"},
                    "action": {"type": "string"},
                    "chiffre": {"type": "integer"},
                    "datetest": {
                        "type": "string",
                        "index": "not_analyzed"
                    },
                    "firewall": {"type": "string"},
                    "host": {"type": "string"},
                    "information": {"type": "string"},
                    "logsource": {
                        "type": "string",
                        "index": "not_analyzed"
                    },
                    "message": {"type": "string"},
                    "numero": {"type": "string"},
                    "operation": {"type": "string"},
                    "portin": {"type": "string"},
                    "portnat": {"type": "string"},
                    "portout": {"type": "string"},
                    "program": {"type": "string"},
                    "reste": {"type": "string"},
                    "tags": {"type": "string"},
                    "timestamp": {"type": "string"},
                    "type": {"type": "string"}
        }}}
    }
}
\end{lstlisting}
\label{lst:mappingresult}

\begin{lstlisting}[style=code,label={lst:mappingput2},caption={Ajouter un template de mapping}]
(DELETE /state-2015.05*)
DELETE /_template/state-log

PUT /_template/state-log
{
"template": "state-*",
"order": 1,
"settings": {"number_of_shards": 5},
"mappings": {
    "syslog": {
        "properties": {
            "@timestamp": {
            "type": "date",
            "format": "dateOptionalTime"
            },
            "@version": {
            "type": "string"
            },
            "action": {
            "type": "string",
            "index": "not_analyzed"
            },
            "chiffre": {
            "type": "integer"
            },
            "datetest": {
            "type": "date",
            "format": "MMM d H:m:s"              
            },
            "firewall": {
            "type": "string"
            },
            "host": {
            "type": "string"
            },
            "information": {
            "type": "string"
            },
            "logsource": {
            "type": "string",
            "index": "not_analyzed"
            },
            "message": {
            "type": "string"
            },
            "reste": {
            "type": "string"
            },
            "tags": {
            "type": "string"
            },
            "timestamp": {
            "type": "string"
            },
            "type": {
            "type": "string"
            }
        }
    }
}
}
\end{lstlisting}
\label{lst:mappingput2}

\section{Code en production}
\label{sec:codeprod}
\lstinputlisting[style=logstash,label={lst:scriptdelindex},caption={Script de suppression d'index}]{../code/remove.sh}

\lstinputlisting[style=logstash,label={lst:scriptkibanaservice},caption={Service simple pour kibana}]{../code/kibana/kibana.service}
Le chemin de l'application est là à titre indicatif, \ipath{/usr/local/bin} est plus 
indiqué, par exemple.


\lstinputlisting[style=logstash,label={lst:scriptfinal},caption={Script de configuration général coté server central (elk2)}]{../code/logstash/logstashfinal.conf}

\lstinputlisting[style=logstash,label={lst:petitscriptfinal},caption={Configuration logstash (elk1)}]{../code/logstash/logstashelk1.conf}

\begin{lstlisting}[style=code,label={lst:configredis},caption={configuration dans /etc/redis/redis.conf}]
{
    daemonize yes
    bind 100.127.255.1  #adresse de l'interface d'écoute

}
\end{lstlisting}
