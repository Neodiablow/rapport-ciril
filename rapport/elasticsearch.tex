\section{Installation}
Avant même de regarder les différentes dépendances nécessaires à l'utilisation de
elasticsearch il est recommandé devérifier que l'on utilise bien la même version
que son compère logstash (vérifier la documentation).

Son installation est sensiblement la même que son comparse logstash puisque ce logiciel 
nécessite également l'installation des dépendances \emph{jruby} et \emph{openjdk-7-jre}
à noter qu'il fonctionne également sur openjdk-8-jre.

Là aussi les paquets debian officiels n'existant pas on utilisera celui fourni par 
elastic.co\footnote{https://download.elastic.co/elasticsearch/elasticsearch/elasticsearch-1.5.1.deb}.
Et ici aussi le paquet est un peu approximatif puisqu'il faut rajouter certains chemin, et ajouter des droits.




Acompleter et revoir
p24-26
 test


\section{Sense}
Sense est un module permettant de faire 

\subsection{Installation}
récupérer le module https://github.com/bleskes/sense puis l'activer dans chromium
chrome://extension developper mode
