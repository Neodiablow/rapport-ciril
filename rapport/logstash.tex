\section{Présentation de Logstash}

\subsection{Pourquoi Logstash?}
Logstash parce que, bien qu'un adminsys compétent soit capable d'analyser
de façon rapide et efficace les logs d'une machine (perl+awk+sed+tail+grep)
Face à des dizaines/milliers de hosts cette méthode ne fonctionne souffre pas
le passage à l'échelle.
De plus il est fréquent, actuellement (cloud, application multicouche) que
que les logs d'une machine ne suffise plus à diagnostiquer un problème.

\section{Installation}
%Sur debian il n'existe pas de paquet deb déjà fait, et le seul prérequis 
%est d'avoir une version de java > 1.7.0\_45
%Pour télécharger et installer
%curl -O https://download.elasticsearch.org/logstash/logstash/logstash-1.4.2.tar.gz
%
%tar zxvf logstash-1.4.2.tar.gz
%
%script before-install.sh
%
%sudo cp -r /home/\$USER/Downloads/logstash-1.4.2 /opt/logstash
%
%sudo mkdir /var/log/logstash
%
%script after-install.sh
%
%cd logstash-1.4.2
%
%
%Création du path
%
%export PATH+=:/opt/logstash
%
%
%
%Dautres méthodes d'installation

%Ensuite un simple git clone https://github.com/elastic/logstash.git
%nous permet de récupérer le dépot stable le plus récent.
%Enfin des scripts d'installation sont disponible
%dans logstash/pkg/debian



Sur Debian jessie il n'existe pas de paquet officiel logstash maintenu. Il existe 
en revanche un paquet tierce chez l'éditeur du logiciel. Le dit paquet n'est pas 
de très bonne facture  puisqu'il nécessite l'ajout de dépendances manuelles ainsi
qu'un rechargement de la configuration de services  \emph{systemctl daemon-reload}.

Les dépendances nécessaires sont \emph{jruby} et \emph{openjdk-7-jre} les mêmes 
que pour elasticsearch.

Une autre manière de réaliser l'installation est d'ajouter les dépots logstash à 
/etc/apt/source.list.d/logstash.list.d/logstash.list

deb https://packages.elasticsearch.org/logstash/1.4/debian stable main

et évidemment la clef qui va bien\footnote{blablabla une clé ça sert à balball}

wget -qO - https://packages.elasticsearch.org/GPG-KEY-elasticsearch | sudo apt-key add -



\section{Utilisation basique}



\section{Mon serveur central}



\section{Monter à l'échelle}
