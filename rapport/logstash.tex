\section{Présentation de Logstash}

\subsection{Pourquoi Logstash?}
Logstash parce que, bien qu'un adminsys compétent soit capable d'analyser
de façon rapide et efficace les logs d'une machine (perl+awk+sed+tail+grep)
Face à des dizaines/milliers de hosts cette méthode ne fonctionne souffre pas
le passage à l'échelle.
De plus il est fréquent, actuellement (cloud, application multicouche) que
que les logs d'une machine ne suffise plus à diagnostiquer un problème.

\subsection{Fonctionnement interne}
L'utilisation de logstash est un pipeline qui s'articule autour de 3 \emph{blocs} 
également appelés \emph{stages} (phase).
\begin{itemize}
    \item   Le bloc : \emph{Inputs} génère des événements à partir des informations reçues
    par logstash en entrée.
    \item   Le bloc : \emph{Filters} modifie, manipule, ces évènements dans logstash
    \item   Le bloc : \emph{Outputs} envoie les évènements de logstash vers leur 
    prochaine destination.
\end{itemize}

Cette façon de fonctionner peut faire penser à \emph{iptables}.

Logstash est un logiciel développé en Jruby\footnote{sauf mention contraire, on supposera
de le reste de ce rapport que logstash est développé en ruby}. Le passage d'une phase 
à l'autre est implémenté via les \emph{SizedQueue} de ruby. Elles sont dimensionnées 
pour contenir 20 éléments\footnote{appelés messages lorsqu'on parle de queues, 
on parle bien ici des \textbf{événements} logstash}, ce n'est pas paramétrable sans 
modifier le code source, c'est un choix délibéré. Ces queues ne sont pas conçues 
pour stocker des données à long terme. On verra plus tard que ce la justifie l'utilisation
d'un \textit{buffer}, comme \emph{Redis}\footnote{voir la sous partie sur la tolérence
de pannes ainsi que le chapitre consacré à Redis}.


Chaque bloc est composé d'une multitude de plugins. Ce sont des modules indépendants
qui peuvent également fonctionner en conjonction les uns des autres.

Il est, par exemple, possible de configure le bloc \emph{inputs} pour utiliser 
plusieurs fois le plugin file (on imagine pour des cas d'utilisation différents) 
et de se servir dans le même temps d'un autre plugin du bloc \emph{inputs} : stdin 
qui prendra typiquement en entrée le clavier.


Il est possible d'implémenter de nouveaux plugins en ruby afin de les ajouter à
notre logstash.

Il existe également un pseudo-bloc qui peut s'insérer dans les autres, ce bloc, \emph{codec} 
permet la de gérer la \textit{représentation des données} c'est à dire qu'un codec
est capable de lire ou d'écrire dans une syntaxe particulière comme par exemple 
rubydebug, collectd ou bien plus intéressant pour nous netflow.

\subsubsection{Tolérance de pannes}
Vos logs sont \textbf{importants}, c'est la raison d'être de Logstash, il ne souhaite 
pas que vous en perdiez le moindre à cause d'un problème réseau ou d'une défaillance
quelconque rendant la destination indisponible.
%cf pipeline.rb et base.rb dans github
Lors d'une indisponibilité, les plugins outputs tentent de renvoyer les événements 
vers leur destination. Si ce n'est pas possible le plugin arrête de lire sa queue 
tant que le message n'a pas pu être envoyé. Par effet domino, une fois la queue 
\textit{filtre => output} remplie, le bloc filtre, ne pouvant plus envoyer de nouveaux 
messages à la queue \textit{FO}. Le plugin du bloc filtre va également retenter 
d'envoyer ses messages, et refuser en attendant de lire les nouveaux arrivant dans 
la queue \textit{input => filtre}.
Si cette dernière venait à se remplir c'est le Logstash tout entier qui refuserait de 
traiter de nouvelles informations. Dans le meilleur des mondes, les expéditeurs de
données se comporterais comme logstash et attendraient patiemment que le problème
se résolve, malheureusement cela n'est pas toujours possible, particulièrement dans
nos problématiques d'où l'importance d'un Redis en amont (ou en aval) afin de 
faire tampon.\footnote{Il existe d'autres outils que Redis, (dont ce n'est pas la fonction
principale) pour réaliser ce travail, ilssont plus adapté mais aussi moins documenté
dans leur utilisation avec logstash.}

\subsubsection{Multithread}
Attention ces informations sont pour le moment, \date{Jeudi 16 Avril}, correctes 
mais sont amenées à changer, notamment concernant les outputs.

Chaque plugin utilise input un \gls{thread}, cela permet d'éviter les engorgements si  
certaines entrées sont plus longues à traiter que d'autres.

En le bloc filtre entier n'utilise par défaut qu'un seul thread, mais il est possible 
d'augmenter le nombre de threads affectés au traitement des filtres avec le \gls{flag}
-w lors du démarrage de Logstash.

À l'heure actuelle le bloc output de logstash ne peut utiliser qu'un seul thread.
Il lit donc sa queue de façon séquentielle.


\section{Installation}
%Sur debian il n'existe pas de paquet deb déjà fait, et le seul prérequis 
%est d'avoir une version de java > 1.7.0\_45
%Pour télécharger et installer
%curl -O https://download.elasticsearch.org/logstash/logstash/logstash-1.4.2.tar.gz
%
%tar zxvf logstash-1.4.2.tar.gz
%
%script before-install.sh
%
%sudo cp -r /home/\$USER/Downloads/logstash-1.4.2 /opt/logstash
%
%sudo mkdir /var/log/logstash
%
%script after-install.sh
%
%cd logstash-1.4.2
%
%
%Création du path
%
%export PATH+=:/opt/logstash
%
%
%
%Dautres méthodes d'installation

%Ensuite un simple git clone https://github.com/elastic/logstash.git
%nous permet de récupérer le dépot stable le plus récent.
%Enfin des scripts d'installation sont disponible
%dans logstash/pkg/debian



Sur Debian jessie il n'existe pas de paquet officiel logstash maintenu. Il existe 
en revanche un paquet tierce\footnote{https://download.elastic.co/logstash/logstash/packages/debian/logstash\_1.4.2-1-2c0f5a1\_all.deb} 
chez l'éditeur du logiciel. Le dit paquet n'est pas de très bonne facture  puisqu'il 
nécessite l'ajout de dépendances manuelles ainsi qu'un rechargement de la configuration 
des services : \emph{systemctl daemon-reload}.



Les dépendances nécessaires sont \emph{jruby} et \emph{openjdk-7-jre} les mêmes 
que pour elasticsearch.

Une autre manière de réaliser l'installation est d'ajouter les dépots logstash à 
/etc/apt/source.list.d/logstash.list.d/logstash.list

deb https://packages.elasticsearch.org/logstash/1.4/debian stable main

et évidemment la clef qui va bien\footnote{blablabla une clé ça sert à balball}

wget -qO - https://packages.elasticsearch.org/GPG-KEY-elasticsearch | sudo apt-key add -



\section{Utilisation basique}

\section{Mon serveur central}



\section{Monter à l'échelle}
