\newglossaryentry{API}{name={API},
                             description={Une API, Application Programming Interface, est une interface permettant d'interagir avec un programme par l'intermédiaire d'"opérations simples". Elles sont en générale la pour faciliter la réutilisation d'un programme plus complexe. C'est parce que openstreetmap à une API que l'on a vu autant de logiciels utilisant ses cartes/tuiles.
}}

\newglossaryentry{cron}{name={cron},
                             description={Cron est un gestionnaire de tâche automatique sur linux, il est extremement répendu.
}}
\newglossaryentry{flag}{name={flag},
                             description={Un drapeau, dans la dénomination logstash, le flag représente une option passé en argument exemple : -w.
}}

\newglossaryentry{cluster}{name={cluster},
                             description={Cluster ou grappe. Un cluster est constitué d'un ou plusieurs nodes qui partagent le même \emph{cluster.name}
}}

\newglossaryentry{document}{name={document},
                             description={Un document, dans la terminologie elasticsearch, est constitué d'un ou plusieurs nodes qui partagent le même \emph{cluster.name}
}}

\newglossaryentry{fulltext}{name={fulltext},
                             description={Un drapeau, dans la dénomination logstash, le flag représente une option passé en argument exemple : -w.
}}

\newglossaryentry{index}{name={index},
                             description={Un index dans la terminologie est ce qui se rapproche d'une base dans le modèle relationnel \emph{classique}. 
}}

\newglossaryentry{lucene}{name={lucene},
                             description={Apache Lucene 
}}
\newglossaryentry{logs}{name={logs},
                             description={Un log c'est une buche !
}}

\newglossaryentry{mapping}{name={mapping},
                             description={Le mapping. Dans la terminologie elasticsearch, un node est processus java 
}}

\newglossaryentry{node}{name={node},
                             description={Node ou noeud est, dans la terminologie elasticsearch, un processus  processus java lucene, c'est une instance d'elasticsearch. 
}}

\newglossaryentry{shard}{name={shard},
                             description={Dans la terminologie elasticsearch, un shard est un index ou un morceau d'index. C'est un index au sens lucene du terme. Il existe deux types de shards, des shards primaires et des shards répliqués. Un shard primaire peut indexer de nouvelles données alors qu'un shard repliqué sert seulement de failover, il peut également être utilisé pour accélérer les réponses aux requêtes.
}}


\newglossaryentry{thread}{name={thread},
                             description={explication thread processus etc.. 
}}

\newglossaryentry{type}{name={type},
                             description={Un type dans la terminologie elasticsearch est l'équivalent d'une table dans le modèle relationnel. Il est équivalent dans la mesure où il possède des champs contenant des valeurs
}}

%\newglossaryentry{BGP}{name={BGP},
%                             description={BGP, Border Gateway Protocle, est un protocole d'échange de routes utilisé sur internet, il permet aux AS (autonomous systems) d'échanger leurs informations.\\
%                             Le plus connu des mauvais usages de BGP s'est produit en février 2008 lorsque Pakistan Telecom a mal appliqué une demande de censure de YouTube, le rendant indisponible pour le monde entier pendant plusieurs heures.\\
%                             Le black-holing BGP est une technique d'endiguement de DDOS souvent utilisée en dernier recours. Pour protéger l'infrastructure réseau (d'un FAI par exemple), on décide que tous les accès à une IP/un range d'IP seront redirigés vers NULL. }}
%\newglossaryentry{CLI}
%{
%	name={CLI},
%	description={Command Line Interface: interface en ligne de commande, se dit d'un programme auquel on accède par l'intermédiaire d'un shell (terminal en ligne de commande). Ces programmes sont en général légers puisqu'ils sont entièrement textuels.}
%}
%
%\newglossaryentry{DDOS}{name={DDOS},
%                        description={Distributed Denial Of Service ou attaque par déni de service distribué. Attaque informatique ayant pour objectif de rendre indisponible un service, un site web etc... en saturant le serveur qui l'héberge (il existe de nombreux type de DDOS différents)}}
%
%\newglossaryentry{firewall}{name={firewall},
%                             description={Pare-feu : système de protection réseau pour un ordinateur ou un réseau d'ordinateur, dans ce projet nous parlerons essentiellement de pare-feu logiciel}}
%
%\newglossaryentry{failover}
%{
%	name={failover},
%	description={Un failover, traduit par passer outre à la panne, est la capacité d'un équipement à basculer automatiquement vers un réseau alternatif ou en veille.}
%}
%
%\newglossaryentry{jumbo frames}{ name={jumbo frames},
%                                description={En réseau, des jumbos frames sont des trames Ethernet dont la longueur dépasse 1500 octets. En général, les jumbo frames peuvent avoir une longueur allant jusqu'à 9000 environ octets.}
%}
%
%\newglossaryentry{hash}
%{
%	name={hash},
%	description={On nomme fonction de hachage une fonction particulière qui, à partir d'une donnée fournie en entrée, calcule une empreinte servant à identifier rapidement, bien qu'incomplètement, la donnée initiale. Les fonctions de hachage sont utilisées en informatique et en cryptographie.}
%}
%
%\newglossaryentry{load balancer}{ name={load balancer},
%                                description={Un répartisseur de charge, ou load balancer, est un ensemble de techniques permettant de distribuer la charge de travail entre plusieurs ordinateurs d'un groupe}
%}
%
%\newglossaryentry{naxsi}
%{
%	name={naxsi},
%	description={Un type de firewall}
%}
%
%\newglossaryentry{OSI}{ name={OSI},
%                                description={Open System Interconnection model est un modèle conceptuel qui caractérise et standardise les communications sur internet. Ce modèle a été standardisé dans l'ISO/IEC 7498-1 (International Organization for Standardization). Il possède 7 couches. 
%                                Réparties comme suit :\\
%    \begin{tabular}{|c|c|c|}
%        \hline
%        Numéro& Couche       & Exemple de protocole\\
%        \hline
%        7&      Applications &  HTTP, NFS, DHCP, BGP, DNS etc... \\
%        \hline
%        6&      Présentation &  MIME, SSL\\ 
%        \hline
%        5&      Session      &  Sockets, SOCKS, NetBIOS\\
%        \hline
%        4&      Transport    & TCP, UDP\\
%        \hline
%        3&      Réseau       & IP4/6, IPsec\\
%        \hline
%        2&      Liaison      & PPP, Token Ring, CRC,\\
%        \hline
%        1&      Physique     & Wifi, Ethernet, CAN, RS-232\\
%        \hline
%    \end{tabular}
%                                }
%}
%
%\newglossaryentry{overhead}
%{
%	name={overhead},
%	description={Coût supplémentaire engendré pour atteindre un but (cette définition est agnostique à la méthode ou la technologie).\\
%    Exemple : Envoyer "Bonjour" à un correspondant par internet (disons sur un chat en ligne), ici la charge utile est "Bonjour"\\
%    Une fois le message enregistré il va être encapsulé de nombreuses fois (TCP, HTTP, conversion en binaire pour la couche physique, etc..)
%    pour être affiché chez notre correspondant. Tout cela prend du temps de calcul, du temps de transport, de la "place" (envoyer toutes
%    ces en-têtes multiplie plusieurs fois la taille du message original). La différence entre le payload (la charge utile originale) et le
%    résultat final est appelé overhead.
%    }
%}
%
%\newglossaryentry{scalabiliser}{ name={scalabiliser},
%                                description={Scalabiliser est un anglicisme courant signifiant mettre à l'échelle. On parle de mise à l'échelle lorsqu'une infrastructure a la capacité de démultiplier sa capacité de traitement. On attend en général que cette mise à l'échelle soit rapide, voire automatisée.}
%}
%
%\newglossaryentry{SSL}
%{
%	name={SSL},
%	description={Secure Socket Layer est un protocole de sécurité SSL qui fonctionne suivant un mode client-serveur. Il permet de satisfaire aux objectifs de sécurité suivants :\begin{itemize}
%	\item l'authentification du serveur.
%	\item la confidentialité des données échangées (ou session chiffrée).
%	\item l'intégrité des données échangées.
%	\end{itemize}
%    voir aussi \gls{TLS}}
%}
%
%\newglossaryentry{TLS}
%{
%	name={TLS},
%	description={Transport Layer Security est une évolution du protocole SSL : cf \gls{SSL}}
%}
%
%\newglossaryentry{wireshark}
%{
%	name={wireshark},
%	description={Wireshark est un analyseur de paquets libre utilisé dans le dépannage et l'analyse de réseaux informatiques, le développement de protocoles, l'éducation et la rétro-ingénierie. Wireshark permet d'analyser les paquets qui transitent sur un réseau en les interceptant.}
%}
%
%
%
%
%
%
%parser
